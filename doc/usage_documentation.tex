\documentclass[a4paper,10pt]{article}
%\documentclass[a4paper,10pt]{scrartcl}

\usepackage[utf8]{inputenc}
\usepackage{german}

\title{Benutzer Handbuch}
\author{}
\date{}

\pdfinfo{%
  /Title    ()
  /Author   ()
  /Creator  ()
  /Producer ()
  /Subject  ()
  /Keywords ()
}

\begin{document}
\maketitle
\tableofcontents
\newpage

\section{Redaktionsansicht}
In diesem Kapitel wird vorausgesetzt, daß die Redaktionsansicht eingestellt ist. Um die Redaktionsansicht einzustellen, kann man auf den Reiter Konto und dann auf den Button ``Redaktionsansicht'' clicken.

\subsection{Register}
\textit{Tabellen: registry\_entries, registry\_names,  registry\_name\_translations, registry\_hierarchies, registry\_entry\_types}\\

\textbf{Reiter Register\slash Thesaurus}\\

\subsubsection{Einträge zusammenführen}
Sobald mindestens zwei der Checkboxen vor den Registereinträgen gecheckt sind erscheint unterhalb der Überschrift der Link ``Registereinträge zusammenführen''. 
Wird dieser Link geclickt, so werden die gecheckten Registereinträge zusammengeführt. \textbf{Der (zeitl.) zuerst gecheckte Registereintrag überlebt}.\\
Alle Referenzen (Schlagworte, Metadatenfelder) die auf die gestorbenen Registereinträge verwiesen hatten, verweisen nun auf den zuerst gecheckten, überlebenden Registereintrag.

\subsubsection{Einträge verschieben}
Click auf das Symbol mit den drei Punkten (``mehr erfahren''). Dann Click auf ``Weiteren bestehenden Elternknoten für diesen Eintrag festlegen'' und in das Formular die ID des Registereintrags schreiben unter welchen der Eintrag verschoben werden soll.\\
Anschliessend kann ``
Zuordung zu diesem Elternknoten entfernen (löscht nicht den Eintrag)'' geclickt werden, wodurch der zu verschiebende Eintrag von seinem ursprünglichen Elternknoten gelöst wird.\\
Durch clicken auf ``
Zuordung zu diesem Elternknoten entfernen (löscht nicht den Eintrag)'' wird nur die Beziehung zum Elternknoten nicht der Eintrag gelöscht!

\subsection{Transkriptupload}
\textit{Tabellen: segments, segment\_translations}\\

\textbf{Reiter Interview, Unterreiter Transkriptupload}\\

Transkripte  können in den Formaten ods, odt, vtt und srt hochgeladen  werden.\\
Um die Zuordnung der Segmente zu den Sprechern sicher zu stellen ist es notwendig, dass für jeden Sprecher die in der Transkriptdatei verwendete Bezeichnung angegeben wird.\\
Sind bereits Mitwirkende (also Interviewt*e, Interviewer*in, etc.) angegeben sollten auf jeden Fall die  Sprecherbezeichnungen abgeglichen werden. Falls noch keine Mitwirkenden angegeben sind können xdiese zugefügt werden. 



\end{document}
